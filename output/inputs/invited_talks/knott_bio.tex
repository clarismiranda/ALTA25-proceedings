Ali Knott is Professor of AI at Victoria University of Wellington. He studied Philosophy and Psychology at Oxford, and did his postgrad and postdoc work at in Edinburgh, in computational linguistics. He moved to New Zealand in 1998, and participated in ALTA during its heady early years.\newline

Ali spent many years researching how language is implemented in the human brain. His main interest was in how language interfaces to the sensory and motor systems, to enable us to talk about what we see and do. He presented his theory about this interface in a book published by MIT Press. He extended the theory in the New Zealand AI company Soul Machines. One application was an embodied model of an 18-month-old toddler, ‘BabyX’.\newline

For the last 10 years, Ali’s main focus has been on the social impacts and governance of AI. He is involved in many international AI policy discussions, mainly through his work as co-lead of a project on Social Media Governance, coordinated by the Global Partnership on AI (now part of the OECD). This work has had several impacts on EU tech legislation, including on provisions about AI content detection (in the AI Act), and provisions to allow vetted researchers access to the largest online platforms (in the Digital Services Act). Ali also participated in the Christchurch Call to eliminate Terrorist and Violent Extremist Content Online, and contributes to the Forum for Information and Democracy. He has advised the New Zealand government on many questions of AI policy, and often looks across the ditch to see how policy is progressing in Australia.\newline